\documentclass[12pt]{article}
\usepackage{pmmeta}
\pmcanonicalname{IntegralEquation}
\pmcreated{2013-03-22 18:03:57}
\pmmodified{2013-03-22 18:03:57}
\pmowner{pahio}{2872}
\pmmodifier{pahio}{2872}
\pmtitle{integral equation}
\pmrecord{8}{40598}
\pmprivacy{1}
\pmauthor{pahio}{2872}
\pmtype{Definition}
\pmcomment{trigger rebuild}
\pmclassification{msc}{45D05}
\pmclassification{msc}{45A05}
\pmrelated{Equation}
\pmdefines{linear integral equation}
\pmdefines{Volterra equation}

% this is the default PlanetMath preamble.  as your knowledge
% of TeX increases, you will probably want to edit this, but
% it should be fine as is for beginners.

% almost certainly you want these
\usepackage{amssymb}
\usepackage{amsmath}
\usepackage{amsfonts}

% used for TeXing text within eps files
%\usepackage{psfrag}
% need this for including graphics (\includegraphics)
%\usepackage{graphicx}
% for neatly defining theorems and propositions
 \usepackage{amsthm}
% making logically defined graphics
%%%\usepackage{xypic}

% there are many more packages, add them here as you need them

% define commands here

\theoremstyle{definition}
\newtheorem*{thmplain}{Theorem}

\begin{document}
An {\em integral equation} involves an unknown function under the \PMlinkescapetext{integral sign}.\, Most common of them is a {\em linear integral equation}
\begin{align}
\alpha(t)\,y(t)+\!\int_a^bk(t,\,x)\,y(x)\,dx = f(t),
\end{align}
where $\alpha,\,k,\,f$ are given functions.\, The function\, $t \mapsto y(t)$\, is to be solved.

Any linear integral equation is \PMlinkname{equivalent}{Equivalent3} to a linear differential equation; e.g. the equation\, $\displaystyle y(t)\!+\!\int_0^t(2t-2x-3)\,y(x)\,dx = 1+t-4\sin{t}$\, to the equation\, $y''(t)-3y'(t)+2y(t) = 4\sin{t}$\, with the initial conditions \,$y(0) = 1$\, and\, $y'(0) = 0$.\\

The equation (1) is of
\begin{itemize}
\item {\em 1st kind} if\, $\alpha(t) \equiv 0$,
\item {\em 2nd kind} if $\alpha(t)$ is a nonzero constant,
\item {\em 3rd kind} else.
\end{itemize}

If both \PMlinkname{limits}{UpperLimit} of integration in (1) are constant, (1) is a {\em Fredholm equation}, if one limit is variable, one has a {\em Volterra equation}.\, In the case that\, $f(t) \equiv 0$,\, the linear integral equation is \PMlinkescapetext{{\em homogeneous}}.\\

\textbf{Example.}\, Solve the Volterra equation\, $\displaystyle y(t)\!+\!\int_0^t(t\!-\!x)\,y(x)\,dx = 1$\, by using Laplace transform.

Using the \PMlinkname{convolution}{LaplaceTransformOfConvolution}, the equation may be written\, $y(t)+t*y(t) = 1$.\, Applying to this the Laplace transform, one obtains\, $\displaystyle Y(s)+\frac{1}{s^2}Y(s) = \frac{1}{s}$,\, whence\, $\displaystyle Y(s) = \frac{s}{s^2+1}$.\, This corresponds the function \,$y(t) = \cos{t}$,\, which is the solution.\\



\PMlinkexternal{Solutions on some integral equations}{http://eqworld.ipmnet.ru/en/solutions/ie.htm} in EqWorld.
%%%%%
%%%%%
\end{document}

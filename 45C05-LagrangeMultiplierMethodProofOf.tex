\documentclass[12pt]{article}
\usepackage{pmmeta}
\pmcanonicalname{LagrangeMultiplierMethodProofOf}
\pmcreated{2013-03-22 15:25:09}
\pmmodified{2013-03-22 15:25:09}
\pmowner{aplant}{12431}
\pmmodifier{aplant}{12431}
\pmtitle{Lagrange multiplier method, proof of}
\pmrecord{6}{37263}
\pmprivacy{1}
\pmauthor{aplant}{12431}
\pmtype{Proof}
\pmcomment{trigger rebuild}
\pmclassification{msc}{45C05}
\pmclassification{msc}{15A42}
\pmclassification{msc}{15A18}

\endmetadata

% this is the default PlanetMath preamble.  as your knowledge
% of TeX increases, you will probably want to edit this, but
% it should be fine as is for beginners.

% almost certainly you want these
\usepackage{amssymb}
\usepackage{amsmath}
\usepackage{amsfonts}

% used for TeXing text within eps files
%\usepackage{psfrag}
% need this for including graphics (\includegraphics)
%\usepackage{graphicx}
% for neatly defining theorems and propositions
%\usepackage{amsthm}
% making logically defined graphics
%%%\usepackage{xypic}

% there are many more packages, add them here as you need them

% define commands here
\begin{document}
Let $g(x,y)=c$ and $f(x,y)=d$.  Taking the derivative of $f$ and $g$ with respect to $t$ gives:


\begin{center}$\displaystyle\frac{\partial f}{\partial t}=\frac{\partial f}{\partial x}x'(t)+\frac{\partial f}{\partial y}y'(t)=0$\end{center}

\begin{center}and\end{center}

\begin{center}$\displaystyle\frac{\partial g}{\partial t}=\frac{\partial g}{\partial x}x'(t)+\frac{\partial g}{\partial y}y'(t)=0$\end{center}

By letting $\vec{r}=x(t)\hat{i}+y(t)\hat{j},$ the partial derivatives can be rewritten as follows:

\begin{center}$\displaystyle\frac{\partial f}{\partial t}=\operatorname{grad} f\cdot \vec{r'};$ \quad $\displaystyle\frac{\partial g}{\partial t}=\operatorname{grad} g\cdot \vec{r'}$\end{center}

This implies that $\operatorname{grad} f \times \operatorname{grad} g = 0,$ thus $\operatorname{grad} f = \lambda \operatorname{grad} g.$  Now this equation can be rewritten as $f_x\hat{i}+f_y\hat{j}=\lambda\left(g_x\hat{i}+g_y\hat{j}\right).$  Since $\mathbb{R}^n\mapsto \mathbb{R},$ this equation can be separated into two new equations:

\begin{center}$\displaystyle f_x=\lambda g_x;\quad f_y=\lambda g_y$\end{center}

Using the above equations, a new function, $F$, can be defined:

\begin{center}$\displaystyle F(x,y,\lambda)=f(x,y)-\lambda g(x,y)$\end{center}

which can be generalized as:

\begin{center}$\displaystyle F(x,y,\lambda)=f(x,y)-\sum_{i=1}^m \lambda_i\left[g_i(x,y)\right].$\end{center}
%%%%%
%%%%%
\end{document}

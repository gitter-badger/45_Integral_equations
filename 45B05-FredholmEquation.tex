\documentclass[12pt]{article}
\usepackage{pmmeta}
\pmcanonicalname{FredholmEquation}
\pmcreated{2013-03-22 15:49:59}
\pmmodified{2013-03-22 15:49:59}
\pmowner{ncrom}{8997}
\pmmodifier{ncrom}{8997}
\pmtitle{Fredholm equation}
\pmrecord{5}{37805}
\pmprivacy{1}
\pmauthor{ncrom}{8997}
\pmtype{Definition}
\pmcomment{trigger rebuild}
\pmclassification{msc}{45B05}

% this is the default PlanetMath preamble.  as your knowledge
% of TeX increases, you will probably want to edit this, but
% it should be fine as is for beginners.

% almost certainly you want these
\usepackage{amssymb}
\usepackage{amsmath}
\usepackage{amsfonts}

% used for TeXing text within eps files
%\usepackage{psfrag}
% need this for including graphics (\includegraphics)
%\usepackage{graphicx}
% for neatly defining theorems and propositions
%\usepackage{amsthm}
% making logically defined graphics
%%%\usepackage{xypic}

% there are many more packages, add them here as you need them

% define commands here
\begin{document}
A \textbf{Fredholm equation of the first kind} is an integral equation of the form
\[
\int_{a}^{b} K(x,y) f(y) dy = g(x), \quad \forall x \in [a,b],
\]
and a \textbf{Fredholm equation of the second kind} is an integral equation of the form
\[
f(x) - \lambda\int_{a}^{b} K(x,y)f(y)dy = g(x), \quad \forall x \in [a, b],
\]
where
\[
Kf(x) = \int_{a}^{b} K(x,y)f(y)dy, \quad \forall x \in [a, b]
\]
is a compact operator in some function space.
%%%%%
%%%%%
\end{document}
